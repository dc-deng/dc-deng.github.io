\documentclass[oneside]{ctexbook}  % oneside参数表示不分左右页
\usepackage{geometry}  % 设置页面格式
\usepackage{amsmath}
\usepackage{amssymb}%some fonts in math environment
\usepackage{mathtools}%dcases
\usepackage{cancel}  % 删除线。有cancel,bcancel等。
\usepackage{graphicx}%include picture
\usepackage{hyperref}%autoref
\usepackage{booktabs}%top rule...
\usepackage[stable]{footmisc}  % 可以在章节标题中作脚注,直接\footnote{}即可

% 页面居中
\geometry{centering}

% % document information
\title{电动}
\date{\today}
\author{dcdeng}

\begin{document}
\maketitle
\chapter{经典电动力学}
\section{静电场的方程式}
库仑定律:静止电荷间的相互作用规律(经验规律).
\begin{equation}
    \mathbf{F }_{j} = \sum_{i\neq j}k\frac{q_{i }q_j}{|\mathbf{r}_j-\mathbf{r}_i|^{2}}\mathbf{e }(\mathbf{r}_i-\mathbf{r}_j), \quad k=\frac{1}{4\pi a_0}(\text{I.U.})
\end{equation}
电力的传递机制:电场.定义为,给一试探电荷$q$, 电场强度$\mathbf{E} = \frac{\mathbf{F }}{q}$.那么由库仑定律,一个静止点电荷电场分布:
\begin{equation}
    \mathbf{E} = \frac{1}{4\pi\epsilon_0}\cdot \frac{q}{r^{2}}\mathbf{e }(\mathbf{r}).
\end{equation}
库伦定律中受力是叠加的,故有叠加原理:
\begin{equation}
    \mathbf{E}(\mathbf{r})=\mathbf{E}_1(\mathbf{r})+\mathbf{E}_2(\mathbf{r}),
\end{equation}
即各个源电荷对总电场的贡献是独立的.对照偏微分方程中解的叠加原理,这性质暗示, $\mathbf{E}$应满足线性的偏微分方程.

已知电荷分布时,
\begin{equation}
    \mathbf{E}(\mathbf{r})=\frac{1}{4\pi\epsilon_0}\int\frac{\rho(\mathbf{r}')\mathbf{e }(\mathbf{r}-\mathbf{r}')}{|\mathbf{r}-\mathbf{r}'|^2}\mathrm{d^3}r'.
\end{equation}
若电荷离散分布, $\rho(\mathbf{r}') = \sum_{i}q_i\delta(\mathbf{r}-\mathbf{r}')$.

下面由库伦定律导出\textbf{静电场}所满足的微分方程.注意到
\begin{equation}
    \nabla\frac{1}{|\mathbf{r}-\mathbf{r}'|}=-\frac{\mathbf{e}(\mathbf{r}-\mathbf{r}')}{|\mathbf{r}-\mathbf{r}'|^2}=-\nabla'\frac{1}{|\mathbf{r}-\mathbf{r}'|},
\end{equation}
$\Rightarrow$
\begin{align}
    \mathbf{E}(\mathbf{r})
     & = \frac{1}{4\pi\epsilon_0}\int_{}^{} \rho(\mathbf{r}')\left( -\nabla\frac{1}{|\mathbf{r}-\mathbf{r}'|} \right) \mathrm{d}^3r' \\
     & =-\nabla\left(\frac{1}{4\pi\epsilon_0}\int_{}^{} \rho(\mathbf{r}') \frac{1}{|\mathbf{r}-\mathbf{r}'|} \mathrm{d}^3r' \right).
\end{align}
引入电势
\begin{equation}
    \phi(\mathbf{r}) = \frac{1}{4\pi\epsilon_0}\int_{}^{} \rho(\mathbf{r}') \frac{1}{|\mathbf{r}-\mathbf{r}'|} \mathrm{d}^3r',
\end{equation}
由
\begin{equation}
    \mathbf{E}=-\nabla\phi,
\end{equation}
立得$\mathbf{E}$的旋度
\begin{equation}
    \nabla\times \mathbf{E}=0.
\end{equation}

\chapter{认知修正暂记}
\begin{enumerate}
    \item 涉及到介质,还是得引入$\mathbf{D}$和$\mathbf{H}$的Maxwell方程组。
    \item 不知道高斯单位制里,不出现$\epsilon$和$\mu$,是怎么处理介质中的电磁场呢?
    \item 电势是对电场标量化的产物。所以改变电荷分布,即改变电场分布,即能改变电势分布。靠近某个电荷附近,电势可能为无穷大。
\end{enumerate}

\end{document}